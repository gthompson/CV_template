\documentclass[margin,line]{res}

\oddsidemargin -.5in
\evensidemargin -.5in
\textwidth=6.0in
\itemsep=0in
\parsep=0in
% if using pdflatex:
%\setlength{\pdfpagewidth}{\paperwidth}
%\setlength{\pdfpageheight}{\paperheight} 

\newenvironment{list1}{
  \begin{list}{\ding{113}}{%
      \setlength{\itemsep}{0in}
      \setlength{\parsep}{0in} \setlength{\parskip}{0in}
      \setlength{\topsep}{0in} \setlength{\partopsep}{0in} 
      \setlength{\leftmargin}{0.17in}}}{\end{list}}
\newenvironment{list2}{
  \begin{list}{$\bullet$}{%
      \setlength{\itemsep}{0in}
      \setlength{\parsep}{0in} \setlength{\parskip}{0in}
      \setlength{\topsep}{0in} \setlength{\partopsep}{0in} 
      \setlength{\leftmargin}{0.2in}}}{\end{list}}

\usepackage{comment}
\usepackage{multicol}

% %%%%%%%% stuff from try/gtpapers.tex
%\documentclass[12pt]{article}
%\usepackage[T1]{fontenc}
%\usepackage[utf8]{inputenc}
%\usepackage{lmodern}
%\usepackage[english]{babel}
%\usepackage[autostyle]{csquotes}
%\RequirePackage{doi}
\usepackage[backend=biber,style=phys,sorting=ydnt]{biblatex}
\addbibresource{try/gtpapers.bib}
\addbibresource{try/gtbookchapters.bib}
%\title{Peer reviewed journal articles}
%\author{Glenn Thompson}
%%%%%%%%%%%

\begin{document}

\name{Glenn Thompson \vspace*{.1in} \hspace{11.2cm} Curriculum vitae} 
\begin{resume}

\vspace{-10pt}
\section{\sc Contact Information}
\vspace{.05in}
\begin{tabular}{@{}p{3in}p{5in}}
School of Geosciences     & {\it Cell:} 907-687-7747  \\            
University of South Florida   &    {\it E-mail:} thompsong@mail.usf.edu\\         
4202 East Fowler Ave., NES107 &  .\\%\it Website: http://seismiclab.web.usf.edu/thompsong\\      
Tampa, FL 33620 USA  & .\\     
\end{tabular}

%\begin{comment}
\section{\sc Key Experience}
Expertise in the design, implementation and integration of seismic monitoring systems. 20 years in geophysical software development. 13 years as an observatory-based network operations seismologist, monitoring regional and volcanic seismicity in Alaska and Montserrat.  Expertise in volcanic seismicity. Experience in managing and communicating volcanic hazards and risks.  Trained in systems analysis and design. Expertise in programming in MATLAB, Perl, PhP and Python, and in managing large seismic databases. Developer of operations software in use at multiple observatories, and a seismic research toolbox downloaded over 1000 times per year.
%\end{comment}

\section{\sc Education}
{\bf Ph.D., Geophysics (Volcano Seismology)} \hfill 1999 \\
University of Leeds, England, UK\\
Thesis title:  ``Modelling of seismo-volcanic sources'' \\
%Advisor: Jurgen Neuberg \\
\\
{\bf M.Sc., Geophysics} \hfill 1995\\
University of Durham, England, UK\\
\\
{\bf B.Sc. Hons., Theoretical Physics and Mathematics} \hfill 1993\\
University of St. Andrews, Scotland, UK
%(GPA 3.8)
%
\begin{comment}
{\bf A Levels} \hfill 1989\\
Rawlins Academy, England, UK\\
Mathematics (A)
Physics (A)
Computer Science (B)
%General Studies
\end{comment}
%
\begin{comment}
{\bf O Levels} \hfill 1987\\
Rawlins Academy, England, UK\\
Mathematics (A)
Physics (A)
Computer Studies (A)
Chemistry (A)
Humanities (B)
English Literature (B)
English Language (C)
Technical Graphics (C)
\end{comment}
%
\section{\sc Professional Appointments}
%
{\bf Research Assistant Professor} \hfill {\bf 2013 - Present} \\
University of South Florida, Tampa, FL USA\\
\begin{comment}
Built the computational infrastructure for the USF Seismology group. Taught classes in volcano
seismology, programming, seismic data analysis and digital signal processing.
Installed seismic and infrasound stations on Masaya volcano and at Kennedy Space Center. Investigating techniques to automatically detect, locate 
and characterize volcano-seismic signals including tremor, swarms, rockfalls, 
pyroclastic flows and lahars.   
\end{comment}
\\
%
{\bf Staff Seismologist (Software Developer)} \hfill {\bf 2006 - 2013} \\
Alaska Volcano Observatory & Alaska Earthquake Center, University of Alaska Fairbanks\\
\begin{comment}
Developed near-real-time, database-driven seismic monitoring software for the Alaska Volcano
Observatory and Alaska Earthquake Center. Investigated algorithms to detect earthquake swarms
at several volcanoes. Participated in volcano and earthquake monitoring, including the 2009
Redoubt eruption. 
\end{comment}
\\
%
{\bf Senior Geophysicist \& Applications Developer} \hfill {\bf 2003 - 2006} \\
British Geological Survey, Nottingham, England \\
\begin{comment}
A wide variety of geophysical surveying projects involving airborne geophysics, resistivity,
refraction seismology, ground penetrating radar. Broad portfolio of software development
projects including writing software to control a robot to scan core samples.
Wrote training manuals for airborne geophysics. Periodically covered for scientists on break
at the Montserrat Volcano Observatory. Further upgraded MVO seismic monitoring systems in 2004.
\end{comment}
\\
%
{\bf Seismic Network Manager \& Deputy Director} \hfill {\bf 2000 - 2003} \\
Montserrat Volcano Observatory, Montserrat, West Indies %(employer: British Geological Survey) 
\\
\begin{comment}
Led the seismic monitoring programme and the Operations Room at MVO. Served as Deputy Director.
Inherited a dysfunctional seismic monitoring programme which jeopardised public safety.
Built a robust seismic monitoring programme, tailored to detecting, locating and characterizing
rockfalls and pyroclastic flows in near-real-time. Merged existing analog and digital networks 
into a single virtual network. Developed alarm, analysis, web-based monitoring and archival 
systems. Recovered, securely archived, and databased all seismic data historically recorded by
MVO and made it available to researchers. Initiated research projects aimed at delivering
a better understanding of volcano-seismicity, and better seismic monitoring tools. 
\end{comment}
\begin{comment}
Real-time, round-the-clock monitoring of the Soufriere Hills Volcano eruption which recommenced in November 1999 and continued until July 2003. Ran the Operations Room by day and 24-hour surveillance of the volcano. Built the seismic monitoring programme, brought in new data acquisition systems, merged the analog and digital seismic networks, developed alarm systems, web-based monitoring tools and a broad suite of data analysis tools. Proposed an infrasound network and a Belham River monitoring network. Seamlessly managed the move to the new observatory in Dec 2002/Jan 2003 without any data loss. Recovered and securely archived and databased all seismic data and initiated numerous projects and collaborations with research scientists. Successfully obtained funding for a new digital network.
Real-time, round-the-clock monitoring of the Soufriere Hills Volcano eruption which recommenced in November 1999 and continued until July 2003. Ran the Operations Room by day and 24-hour surveillance of the volcano. Overhauled the entire seismic monitoring programme and refocused it on novel techniques to rapidly detect and characterize volcanic hazards.
\end{comment}
\\
%
{\bf Postdoctoral Investigator (Software Developer)} \hfill {\bf 1998 - 2000} \\
Alaska Volcano Observatory, University of Alaska Fairbanks\\
\begin{comment}
Supervisor: Stephen R. McNutt\\
Developed the first web-based seismic monitoring system for volcanoes. Featured near-real-time
spectrograms, reduced displacement and helicorder plots, and a web-configurable tremor alarm
system. This has remained a core AVO monitoring tool since 1998 and inspired spin-offs at other
observatories. Participated in volcano and
earthquake monitoring, including the Shishaldin eruption in 1999. Investigated different types
of tremor at Shishaldin volcano.
\end{comment}
\\
%
{\bf Systems Analyst/Programmer} \hfill {\bf 1997 - 1998}\\
TNT Express Worldwide, Atherstone, England, UK\\
\begin{comment}
Three month intensive training course in systems analysis, software design and programming
Assigned to the re-engineering team thereafter.
\end{comment}
\\
%
{\bf Seismologist} \hfill {\bf Summer 1996}\\
Montserrat Volcano Observatory, Montserrat, West Indies \\
% (employer: British Geological Survey)
\\
%
{\bf Summer Student} \hfill {\bf Summer 1992}\\
CERN, Meyrin, Geneva, Switzerland
\begin{comment}
Three month intensive training course in particle physics.
Computer simulation of muon detectors.
\end{comment}
\\
%
\begin{comment}
{\bf Electronics Technician} \hfill {\bf Summer 1989 & 1990}\\
Druck Ltd., Groby, Leics, UK
%\begin{comment}
Testing and calibration of pressure transducers.
%\end{comment}
\end{comment}

%Books, Edited Volumes, Refereed Journal Articles, Book Chapters, Conference Proceedings, Book Reviews, Manuscripts in Submission (give journal title), Manuscripts in Preparation, Web-Based Publications, Other Publications (this section can include non-academic publications, within reason). Please note that forthcoming publications ARE included in this section. If they are already in the printing stage, with the full citation and page numbers available, they may be listed the same as other published publications, at the very top since their dates are furthest in the future. If they are in press, they can be listed here with ��in press�� in place of the year.

\section{\sc Publications}
\documentclass[12pt]{article}
%\usepackage[T1]{fontenc}
%\usepackage[utf8]{inputenc}
%\usepackage{lmodern}
%\usepackage[english]{babel}
%\usepackage[autostyle]{csquotes}
%\RequirePackage{doi}
\usepackage[backend=biber,style=phys,sorting=ydnt]{biblatex}
\addbibresource{gtpapers.bib}
\addbibresource{gtbookchapters.bib}
%\title{Peer reviewed journal articles}
%\author{Glenn Thompson}

\begin{document}
\nocite{*}
%\printbibliography[title={Refereed Journal Articles}]
\printbibliography[heading=none]
\\
.\\
Conference papers: 65 (17 as first author)\\ % see gtconferencepapers.tex
Technical reports: 31 (25 as first author)\\ % see gtreports.tex


\end{document}


\section{\sc Publications - Book Chapters}

	(2) {\bf Thompson G.} (in press). Seismic Monitoring of Volcanoes. In: Michael Beer, Ioannis A. Kougioumtzoglou, Edoardo Patelli, and Ivan Siu-Kui Au (Eds.) Encyclopedia of Earthquake Engineering. Springer-Verlag Berlin Heidelberg.
	
	(1) McNutt, S.R., {\bf Thompson, G.}, Fee, D., Johnson, J.B., and De Angelis, S. (in press). Seismic and Infrasonic Monitoring, Encyclopedia of Volcanoes, 2nd edition.

\section{\sc Publications - Refereed Journal Articles}

	(17) DeRoin, N., McNutt, S.R., and {\bf Thompson, G.} (2015). Duration-amplitude relationships of volcanic tremor and earthquake swarms preceding and during the 2009 eruption of Redoubt Volcano, Alaska. Journal of Volcanology and Geothermal Research, doi:10.1016/j.volgeores.2015.01.003
	
	(16) McNutt, S.R., {\bf Thompson, G.}, West, M.E., Fee, D., Stihler, S., and Clark, E. (2013). Local seismic and infrasound observations of the 2009 explosive eruptions of Redoubt Volcano, Alaska. Journal of Volcanology and Geothermal Research, 259, 63-76. doi:10.1016/j.jvolgeores.2013.03.016

	(15) Buurman, H., West, M.E., and {\bf Thompson, G.} (2012). The seismicity of the 2009 Redoubt eruption. Journal of Volcanology and Geothermal Research, 259, 16-30. doi:10.1016/j.jvolgeores.2012.04.024

	(14) Schaefer, J.R., ed., (2012), The 2009 eruption of Redoubt Volcano, Alaska, with contributions by Bull, K., Cameron, C., Coombs, M., Diefenbach, A., Lopez, T., McNutt, S., Neal, C., Payne, A., Power, J., Schneider, D., Scott, W., Snedigar, S., {\bf Thompson, G.}, Wallace, K., Waythomas, C., Webley, P., and Werner, C.: Alaska Division of Geological and Geophysical Surveys Report of Investigation 2011-5, 45 p., available at http://www.dggs.alaska.gov/pubs/id/23123.

	(13) Miller, V., Voight, B., Ammon, C., Shalev, E., and {\bf Thompson, G.} (2010). Seismic expression of magma-induced crustal strains and localized fluid pressures during initial eruptive stages, Soufriere Hills Volcano, Montserrat. Geophysical Research Letters, 37(19) 

	(12) {\bf Thompson, G.}, and West, M.E. (2010). Real-time Detection of Earthquake Swarms at Redoubt Volcano, 2009. Seismological Research Letters, 81(3), 505-513. doi:10.1785/gssrl.81.3.505

	(11) Luckett, R., Baptie, B., Ottemoller, L., and {\bf Thompson, G.} (2007). Seismic Monitoring of the Soufriere Hills Volcano, Montserrat. Seismological Research Letters, 78(2), 192-200. doi:10.1785/gssrl.78.2.192

	(10) Taron, J., Elsworth, D., {\bf Thompson, G.}, and Voight, B. (2007). Mechanisms for rainfall-concurrent lava dome collapses at Soufriere Hills Volcano, 2000-2002. Journal of Volcanology and Geothermal Research, 160(1-2), 195-209. doi:10.1016/j.jvolgeores.2006.10.003

	(9) Jaquet, O., Carniel, R., Sparks, S., {\bf Thompson, G.}, Namar, R., and Dicecca, M. (2006). DEVIN: A forecasting approach using stochastic methods applied to the Soufriere Hills Volcano. Journal of Volcanology and Geothermal Research, 153(1-2), 97-111. doi:10.1016/j.jvolgeores.2005.08.013 

	(8) Langer, H., Falsaperla, S., Powell, T., and {\bf Thompson, G.} (2006). Automatic classification and a-posteriori analysis of seismic event identification at Soufriere Hills Volcano, Montserrat. Journal of Volcanology and Geothermal Research, 153(1-2), 110. doi:10.1016/j.jvolgeores.2005.08.012

	(7) Carn, S. A., Watts, R. B., {\bf Thompson, G.}, and Norton, G. E. (2004). Anatomy of a lava dome collapse. Journal of Volcanology and Geothermal Research, 131, 241-264.

	(6) Elsworth, D., Voight, B., {\bf Thompson, G.}, and Young, S. R. (2004). Thermal-hydrologic mechanism for rainfall-triggered collapse of lava domes. Geology, 32(11), 969-972. doi:10.1130/G20730.1

	(5) Edmonds, M., Oppenheimer, C., Pyle, D. M., Herd, R. A., and {\bf Thompson, G.} (2003). SO2 emissions from Soufriere Hills Volcano and their relationship to conduit permeability, hydrothermal interaction and degassing regime. Journal of Volcanology and Geothermal Research, 124(1-2), 23-43. doi:10.1016/S0377-0273(03)00041-6

	(4) Langer, H., Falsaperla, S., and {\bf Thompson, G.} (2003). Application of Artificial Neural Networks for the classification of the seismic transients at Soufriere Hills Volcano, Montserrat. Geophysical Research Letters, 30(21), 1-5. doi:10.1029/2003GL018082

	(3) Matthews, A. J., Barclay, J., Carn, S., {\bf Thompson, G.}, Alexander, J., Herd, R., and Williams, C. (2002). Rainfall-induced volcanic activity on Montserrat. Geophysical Research Letters, 29(13). doi:10.1029/2002GL014863

	(2) Jolly, A. D., {\bf Thompson, G.}, and Norton, G. (2002). Locating pyroclastic flows on Soufriere Hills Volcano, Montserrat, West Indies, using amplitude signals from high dynamic range instruments. Journal of Volcanology and Geothermal Research, 118(3-4), 299-317. doi:10.1016/S0377-0273(02)00299-8

	(1) {\bf Thompson, G.}, McNutt, S. R., and Tytgat, G. (2002). Three distinct regimes of volcanic tremor associated with the eruption of Shishaldin Volcano, Alaska 1999. Bulletin of Volcanology, 64(8), 535-547. doi:10.1007/s00445-002-0228-z


\section{\sc Publications - Peer-reviewed Open-File Reports}

	(3) Gunn, D. A., Jackson, P. D., {\bf Thompson, G.} (2005), Fine Scale Structure of Core by Novel Resistivity Imaging. BGS Report No. IR/05/065.

	(2) Gunn, D. A., Nelder L. M., Pearson, S., {\bf Thompson, G.}, Carney, J. N., Wallis, H., (2005), Investigating shear wave methods to characterize sand and gravel thicknesses at Holme Pierrepont, Nottingham. BGS Report No. IR/03/116.

	(1) Cuss, R. J., {\bf Thompson, G.}, (2003), Ground penetrating radar investigation of the fissuring of the A690 in Houghton-le-Spring , Commissioned Report CR/03/301R, Urban Geoscience and Geological Hazards Programme, British Geological Survey.

\begin{comment}	
\section{\sc Theses}

	{\bf Thompson, G.}, (1999), Modelling of seismo-volcanic sources. Ph.D. Thesis, University of Leeds, UK.

	{\bf Thompson, G.}, (1994), Forward modelling of bottom-simulating reflectors at the Makran accretionary prism. M.Sc. Dissertation, University of Durham, UK.
\end{comment}

\begin{comment}
\section{\sc Publications - Conference Abstracts}

	%%George, O., Latchman, J.L., {\bf Thompson, G.}, Connor, C.B. and Malservisi, R. (2015). Improving Seismic Event Locations on Dominica Using the HypoDD algorithm to Enhance Volcanic Hazard Assessments, ... 2015.

	(47) {\bf Thompson, G.}, (2014). Towards a Comprehensive Catalog of Volcanic Seismicity. EOS Trans., AGU., ..., Fall Meet. Suppl., Abstract ..., December 2014.

	(46) Smith, C., McNutt, S.R, and {\bf Thompson, G.} (2014). Explosion Quakes: The 2007 Eruption of Pavlof. EOS Trans., AGU., ..., Fall Meet. Suppl., Abstract ..., December 2014.

	(45) McFarlin, H.L, Christensen, D.H., {\bf Thompson, G.}, McNutt, S.R., Ryan, J.C., Ward, K.M., Zandt, G., and West, M.E. (2014). Receiver Function Analyses of Uturuncu Volcano (Bolivia) and Lastarria/Cordon Del Azufre Volcanoes (Chile). EOS Trans., AGU., ..., Fall Meet. Suppl., Abstract ..., December 2014.

	(44) {\bf Thompson, G.}, and McNutt, S.R. (2014). Banded tremor at Soufriere Hills Volcano, Montserrat. Seismo. Soc. Am. Annual Meeting, Anchorage, 30 April - 2 May 2014.

	(43) McFarlin, H., Christensen, D., and {\bf Thompson, G.} (2014). Receiver Function Analyses of Uturuncu Volcano, Bolivia. Seismo. Soc. Am. Annual Meeting, Anchorage, 30 April - 2 May 2014.

	(42) {\bf Thompson, G.}, West, M. E., (2010), Real-time Tracking of Earthquake Swarms at Redoubt Volcano, 2009. Seismo. Soc. Am. Annual Meeting, Portland, 21-23 April 2010.

	(41) {\bf Thompson, G.}, West, M. E., (2009), Alarm systems detect volcanic tremor and earthquake swarms during Redoubt eruption, 2009. EOS Trans., AGU, 90 (52), Fall Meet. Suppl., Abstract V43A-2215, December 2009.

	(40) Miller, V., Ammon, C., Voight, B., and {\bf Thompson, G.} (2006). Precise hypocenter location of high-frequency-onset earthquakes, during the initial stages of activity at Soufriere Hills Volcano, Montserrat. AGU Fall Meeting Abstracts, 1, 0586. 

	(39) Elsworth, D., Voight, B., Taron, J., {\bf Thompson, G.}, Vinciguerra, S., and Simmons, J. (2005). Gravitational collapse of lava domes triggered by volcanic fluids. AGU Fall Meeting Abstracts, 1, 07. 

	(38) Mattioli, G.S., Voight, B., Linde, A.T., Sacks, I.S., Watts, P., Hidayat, D., Young, S.R.,  Widiwijayanti, C., Shalev, E., Malin, P.E., and {\bf Thompson, G.} (2005). The CALIPSO borehole project at Soufriere Hills Volcano, Montserrat, BWI: Status and scientific overview of prodigious dome collapse of July 2003. AGU Spring Meeting Abstracts, 1, 05. 

	(37) Voight, B., Mattioli, G.S., Linde, A.T., Sacks, I.S., Young, S.R., Malin, P., Shalev, E., Hidayat, D., Elsworth, D., Widiwijayanti, C., Miller, V., McWhorter, N., Schleigh, B., Johnston, W., Sparks, R.S.J., Neuberg, J., Bass, V., Dunkley, P., Herd, R., Jolly, A.D., Norton, G., Syers, T., {\bf Thompson, G.}, Williams, P., Williams, D., Clarke, A., CALIPSO Borehole Monitoring Project at Soufriere Hills Volcano, Montserrat, BWI: Overview, and Response of Magma Reservoir to Prodigious Dome Collapse. Scientific conference, Montserrat, July 2005.

	(36) Miller V., Ammon, C., Voight, B., {\bf Thompson, G.}, Precise hypocenter location of high-frequency-onset earthquakes and changing stress conditions beneath Soufriere Hills volcano, Montserrat. Scientific conference, Montserrat, July 2005.

	(35) Voight, B., Mattioli, G., Linde, A., Sacks, I., Young, S., Malin, P., Shalev, E., Hidayat, D., Elsworth, D., Widiwijayanti, C., Miller, V., McWhorter, N., Schleigh, B., Johnston, W., Sparks, R., Neuberg, J., Bass, V., Dunkley, P., Herd, R., Jolly, A., Norton, G., Syers, T., {\bf Thompson, G.}, Williams, C., Williams, D., Clarke, A. (2005). CALIPSO Borehole Monitoring Project at Soufriere Hills Volcano, Montserrat, BWI: Overview, and Response of Magma Reservoir to Prodigious Dome Collapse. Soufrière Hills Volcano -- Ten-Years On Scientific Conference. (Jul 2005).

	(34) {\bf Thompson, G.} (2005). Advances in seismic monitoring at the Montserrat Volcano Observatory 2000-2003. Soufrière Hills Volcano -- Ten-Years On Scientific Conference, July 2005.

	(33) Mattioli, G.S., Young, S.R., Linde, A.T., Sacks, I.S., Malin, P.E., Shalev, E., Hidayat, D., Elsworth, D., Widiwijayanti, C., Miller, V., and {\bf Thompson, G.} (2004). Calipso borehole instrumentation at a Bezymianny-like andesite Volcano: Calipso project at Soufriere Hills Volcano, Montserrat., 82. 

	(32) Mattioli, G., Voight, B., Young, S., Linde, A., Sacks, I., Malin, P., Shalev, E., Hidayat, D., Elsworth, D., Widiwijayanti, C., Miller, V., McWhorter, N., Schleigh, B., Johnston, W., Sparks, R., Neuberg, J., Bass, V., Dunkley, P., Herd, R., Jolly, A., Norton, G., Syers, T., {\bf Thompson, G.}, Williams, P., Williams, D., Clarke, A. (2004). CALIPSO project at Soufriere Hills Volcano, Montserrat: Prototype PBO instrumentation installed and captures massive dome collapse of July 2003. IAVCEI General Assembly. (Nov 2004).

	(31) Miller, V., {\bf Thompson, G.}, Voight, B., and Ammon, C. (2004). Precise hypocenter location of high-frequency-onset earthquakes, tomography, and changing stress conditions beneath Soufriere Hills Volcano, Montserrat. AGU Fall Meeting Abstracts, 1, 05. 

	(30) Shannon, J., G. Bluth, G. Ryan, A.D. Jolly, G. Thompson, and M. Edmonds (2004). Short-term fluctuations in SO2 emission from automated DOAS measurement on Montserrat, IAVCEI, Pucon, Chile, 2004.

	(29) Voight B., Mattioli G.S., Linde A.T., Sacks I.S., Watts P., Hidayat D., Young S.R., Widiwijayanti C., Shalev E., Malin P.E., Elsworth D., Williams P., Van Boskirk E., Thompson G., Syers T., Sparks R.S.J., Schleigh B., Norton G., Neuberg J., Miller V., McWhorter N., Johnston W., Dunkley P., Clarke A.B., Bass V., Collapse of lava dome on Montserrat, July 2003 with unique CALIPSO geophysical measurement of pyroclastic flows and PF-generated tsunami waves, IAVCEI General Assembly, Pucon, Chile, 2004.

	(28) Langer, H., S. Falsaperla, T. Powell and G. Thompson (2004). Reclassification of seismic transients at Soufriere Hills Volcano, Montserrat, and applications of Artificial Neural Networks. EGU General Assembly, Nice 2004, EGU04-A-04026, April 2004.

	(27) Voight, B., Mattioli, G.S., Linde, A.T., Sacks, I.S., Young, S.R., Malin, P.E., Shalev, E., Hidayat, D., Elsworth, D., Widiwijayanti, C., and {\bf Thompson, G.} (2004). CALIPSO borehole monitoring project at Soufriere Hills Volcano, Montserrat, BWI: Overview, and response of magma reservoir to prodigious dome collapse. AGU Fall Meeting Abstracts, 1, 03. 

	(26) Shannon, J., G. Bluth, M. Edmonds, and G. Thompson (2003). Volcanic SO2 emissions vs. seismicity - July 2002 LP swarm. EOS Trans., AGU, 84 (46), Fall Meet. Suppl., Abstract V52B-0435, December 2003. 

	(25) Dunkley, P. Voight, B., Edmonds, M., Herd, R., Strutt, M., {\bf Thompson, G.}, Bass, V., Aspinall, W.P., Neuberg, J., and Sparks, R. (2003). The rise and fall of the Soufrière Hills Volcano lava dome, Montserrat, BWI, July 2001-July 2003: Science, hazards, and volatile public perceptions. EOS Trans., AGU, 84 (46), Fall Meet. Suppl., Abstract V51F-0342

	(24) Hidayat, D., Voight, B., Mattioli, G., Young, S.R., Linde, A.T., Sacks, I.S., Malin, P.E., Shalev, E., Elsworth, D., Widiwijayanti, C., and {\bf Thompson, G.} (2003). Seismo-acoustics, VLP and ULP signals, and other comparisons of surface broadband and CALIPSO borehole data at Soufriere Hills Volcano, Montserrat, BWI. EOS Trans., AGU, 84 (46), Fall Meet. Suppl.,

	(23) Voight, B., G.S. Mattioli, S.R. Young, A.T. Linde, I.S. Sacks, P.E. Malin, E. Shalev, D. Hidayat, D. Elsworth, C. Widiwijayanti, V. Miller, R.S.J. Sparks, J. Neuberg, V. Bass, P. Dunkley, M. Edmonds, R. Herd, A. Jolly, G. Norton, and G. Thompson (2003).  CALIPSO Borehole Instrumentation Project at Soufriere Hills volcano, Montserrat, BWI: Overview and prospects. EOS Trans., AGU, 84 (46), Fall Meet. Suppl., Abstract V51J-0409, 2003.

	(22) Young, S.R., B. Voight, G.S. Mattioli, A.T. Linde, I.S. Sacks, P.E. Malin, E. Shalev, D. Hidayat, D. Elsworth, R.S. Sparks, J. Neuberg, P.N. Dunkley, G. E. Norton, R.A. Herd, M. Edmonds, G. Thompson, A. Jolly and V. Bass, 2003. Linking surface activity to the deep volcanic plumbing sustem: the CALIPSO borehole observatory project on Montserrat. EOS Trans., AGU, 84(46), Fall Meet. Suppl., Abstract V51J-0409, 2003.

	(21) Langer H., Falsaperla S., Powell T., {\bf Thompson, G.} (2003). Classification of seismic transients recorded on Soufriere Hills volcano, Montserrat, using Artifical Neural Networks. 13th ESC-WG Annual meeting, Pantelleria, Italy, 23-28 September 2003.

	(20) Langer, H., Falsaperla, S., and {\bf Thompson, G.} (2003). Automatic identification of seismic transients recorded on Soufriere Hills Volcano, Montserrat. EGS-AGU-EUG Joint Assembly, 1. pp. 3531. 

	(19) Neuberg, J., Edmonds, M., Herd, R., {\bf Thompson, G.}, Green, D., Powell, T., and Jolly, A. (2003). Multi-parameter Monitoring on Montserrat: Concepts, Megabytes of Data and Results. IUGG General Assembly, Sapporo, Japan, 30 June - 11 July 2003.
	
	(18) Young, S., Voight, B., Edmonds, M., Herd, R., {\bf Thompson, G.}, and Mattioli, G. (2003). Investigation of cyclic activity at Soufriere Hills volcano, Montserrat, through multiple monitoring techniques. IUGG General Assembly, Sapporo, Japan, 30 June - 11 July 2003.

	(17) Jolly A.D., Herd, R. A., Norton, G. E., {\bf Thompson, G.}, Bass, V. A. (2003). Location and duration-size distribution of dome failure events at the Soufriere Hills Volcano, Montserrat, West Indies. IUGG General Assembly, Sapporo, Japan, 30 June - 11 July 2003.

	(16) {\bf Thompson, G.}, Dunkley, P.N., Edmonds, M., Herd, R. A. (2003). Recent advances in volcano monitoring at the Montserrat Volcano Observatory, Seismo. Soc. Am. Annual Meeting, Puerto Rico, 29 April - 3 May, 2003. 

	(15) {\bf Thompson, G.} (2002). Volcano-seismic monitoring of the Soufriere Hills Volcano, Montserrat. 12th ESC-WG Annual Meeting, Montserrat, September 2002. 

	(14) Elsworth, D., Voight, B., Calder, E.S., Edmonds, M., Herd, R., Norton, G., Syers, T., {\bf Thompson, G.}, Watts, R., and Young, S.R. (2002). Some Models of Rainfall-triggered Collapse of Lava Domes in Potentially Gas-effusive Environments, May 2002.

	(13) Voight, B., Young, S.R., Baptie, B.J., Bass, V., Duffell, H., Dunkley, P.N., Edmunds, M., Herd, R.A., Jolly, A.D., Norton, G. and {\bf Thompson, G.} (2001). Multiparameter measurements at Montserrat and their interpretation: Honoring the memory of Bruno Martinelli. AGU Fall Meeting Abstracts, 1, 02. 

	(12) Jolly, A.D., {\bf Thompson, G.}, and Norton, G.E. (2001). Locating Pyroclastic Flows on Soufriere Hills Volcano, Montserrat, West Indies, Using Amplitude Signals From High Dynamic Range Instruments. EOS Trans., AGU, 82(47), December 2001.

	(11) Voight, B., Young, S., Baptie, B. Bass, V., Duffell, H., Dunkley, P., Edmonds, M., Herd, R., Jolly, A., Norton, G., Syers, T., {\bf Thompson, G.}, Williams, C., Williams, D. (2001). EOS Trans., AGU, 82(47), Fall Meet. Suppl., 2001.

	(10) {\bf Thompson, G.}, and Jolly, A.D. (2001). Detecting switches in dome growth direction by mapping rockfall activity at Soufriere Hills Volcano, Montserrat, 11th ESC-WG Annual meeting, Tenerife, September 2001.
	
	(9) {\bf Thompson, G.}, Powell, T.W., and Neuberg, J. (2001). Cataloguing banded tremor episodes at the Soufriere Hills Volcano, Montserrat, 1996-2001. EU Workshop on Calderas, Vesuvius Volcano Observatory, Italy, May 2001. 

	(8) Jolly, A. D., Young, S. R., Cabey, L., {\bf Thompson, G.} (1999). The Reduced Displacement of large explosive eruptions at Soufriere Hills volcano, Montserrat, West Indies, using broadband seismic data. EOS Trans., AGU, 80, 1145, December 1999

	(7) {\bf Thompson, G.}, McNutt, S. R., Mann, D., Bower, G. R., (1999). Monitoring and analysis of volcanic tremor reduced displacement and spectra associated with eruptions of Shishaldin Volcano, April 1999. EOS Trans., AGU, December 1999. 

	(6) {\bf Thompson, G.}, Benoit, J., Lindquist, K., Hansen, R., and McNutt, S. (1998). Near-realtime WWW-based monitoring of Alaskan Volcanoes: The IceWeb system. EOS, Transactions, American Geophysical Union, 79(45), F957. 

	(5) {\bf Thompson, G.} and Neuberg, J. (1998). Amplitude modelling of long period seismic phases at Stromboli and estimation of source parameters. EOS Trans., AGU, December 1998.

	(4) {\bf Thompson, G.} and Neuberg, J. (1997). Wavenumber-frequency modelling of long period seismic phases at Stromboli Volcano. 7th ESC-WG Annual Meeting, Ambleside, UK, September 1997. 

	(3) Neuberg J., {\bf Thompson, G.}, and Luckett, R. (1996). Models for Strombolian eruptions inferred from broadband data. Annales Geophysicae, Supplement 1 to Vol 14, C281, 1996.

	(2) {\bf Thompson, G.} (1996). Banded tremor at Montserrat, July-August 1996, Volcanic Studies Group, The Geological Society, Burlington House, London UK, 27 November 1996.

	(1) Neuberg, J., and {\bf Thompson, G.}, (1995). A model of Strombolian eruptions inferred from seismic broadband data. EOS Trans., AGU, 76, 46, 658, December 1995.
\end{comment}

%\section {\sc Publications - Submitted}

%\section {\sc Publications - In Preparation}\\

\begin{comment}
\section{\sc Publications - Technical \& Scientific Reports}

	(27) {\bf Thompson, G.} (2013). Web-based seismic spectrograms at the GI. GISEIS Technical Whitepaper 2013-03, Geophysical Institute, University of Alaska Fairbanks.

	(26) {\bf Thompson, G.} (2013). AEIC Web Presence Project. GISEIS Technical Whitepaper 2013-02, Geophysical Institute, University of Alaska Fairbanks.

	(25) {\bf Thompson, G.} (2013). Mirroring the AVO event catalog from AQMS to Antelope. GISEIS Technical Whitepaper 2013-01, Geophysical Institute, University of Alaska Fairbanks.

	(24) {\bf Thompson, G.} (2011). Optimizing detection of earthquakes swarms in the presence of noise, using Antelope. GISEIS Technical Whitepaper 2011-01, Geophysical Institute, University of Alaska Fairbanks.

	(23) {\bf Thompson, G.} (2009). IceWeb2. GISEIS Technical Whitepaper 2009-01, Geophysical Institute, University of Alaska Fairbanks.

	(22) {\bf Thompson, G.} Robinson, M., (2008). Configuring an iMac as an Earthquake Notification System. GISEIS Technical Whitepaper 2008-06, Geophysical Institute, University of Alaska Fairbanks.

	(21) {\bf Thompson, G.} (2008). User Guide to the iMac Earthquake Notification. GISEIS Technical Whitepaper 2008-05, Geophysical Institute, University of Alaska Fairbanks.

	(20) {\bf Thompson, G.} (2008). Delivery of Earthquake Notification Systems to Emergency Managers in Alaska. GISEIS Technical Whitepaper 2008-04, Geophysical Institute, University of Alaska Fairbanks.

	(19) {\bf Thompson, G.}, Robinson, M. (2008). The Trans-Alaska Pipeline ShakeMap system. GISEIS Technical Whitepaper 2008-03, Geophysical Institute, University of Alaska Fairbanks.

	(18) {\bf Thompson, G.}, Martirosyan, A., Robinson, M., Hansen, R. (2008). The AEIC ShakeMap system. GISEIS Technical Whitepaper 2008-02, Geophysical Institute, University of Alaska Fairbanks.

	(17) {\bf Thompson, G.}, La Fevers, M. (2008). Revised procedures for maintaining the AEIC master stations database. GISEIS Technical Whitepaper 2008-01, Geophysical Institute, University of Alaska Fairbanks.

	(16) {\bf Thompson, G.}, Robinson, M. (2007). The AEIC Antelope-QDDS Interface. GISEIS Technical Whitepaper 2007-02, Geophysical Institute, University of Alaska Fairbanks.

	(15) {\bf Thompson, G.}, La Fevers, M. (2007). Maintaining the AEIC master stations database. GISEIS Technical Whitepaper 2007-01, Geophysical Institute, University of Alaska Fairbanks.
	
	(14) {\bf Thompson, G.} (2004). Rebuilding MVO seismic monitoring, January - February 2004. MVO Open File Report 04/04, Montserrat Volcano Observatory, 2004.

	(13) Dunkley, P.N., Norton, G.E., Hards, V.L., Ryan, G.A., Jolly, A.D., {\bf Thompson, G.} (2004). Summary of volcanic activity from May 2003 to February 2004. MVO Open File Report 04/02, Montserrat Volcano Observatory, 2004.

	(12) {\bf Thompson, G.} (2003). The role of the Software Engineer within the seismic monitoring programme. MVO Open File Report 03/03, Montserrat Volcano Observatory, 2003.

	(11) {\bf Thompson, G.} (2003). Moving the seismic monitoring from Mongo Hill to Flemings. MVO Open File Report 03/02, Montserrat Volcano Observatory, 2003.

	(10) Dunkley, P.N., Edmonds, M, Herd, R.A., {\bf Thompson, G.} (2003). Summary of volcanic activity and monitoring data, with particular emphasis on the second phase of dome building November 1999 to June 2003. MVO Open File Report 03/01, Montserrat Volcano Observatory, 2003.

	(9) {\bf Thompson, G.} (2002). Seismic software at MVO, January 2002. MVO Open File Report 02/01, Montserrat Volcano Observatory, 2002.

	(8) {\bf Thompson, G.}, McNutt, S.R. (2002). Near-real-time web-based volcano-seismic monitoring: The IceWeb system. GISEIS Technical Whitepaper 2002-01, Geophysical Institute, University of Alaska Fairbanks.

	(7) {\bf Thompson, G.} (2001). An overview of banded tremor at the Soufriere Hills Volcano, 1996-2001. MVO Open File Report 01/02, Montserrat Volcano Observatory, 2001.

	(6) {\bf Thompson, G.}, Dunkley, P.N., Edmonds, M., Herd, R.A., Jolly, A.D. (2001). The 29 July 2001 collapse. MVO Special Report 9, Montserrat Volcano Observatory, 2001.

	(5) {\bf Thompson, G.} (2000). Upgrading the MVO seismic data acquisition and analysis systems and collaborating with SRU, October 2000. MVO Open File Report 00/03, Montserrat Volcano Observatory, 2000.

	(4) {\bf Thompson, G.} (2000). Review of MVO Seismic Monitoring, August 2000. MVO Open File Report 00/02, Montserrat Volcano Observatory, 2000. 

	(3) {\bf Thompson, G.}, Watts, R., Carn, S., Norton, G. Nye, R., Voight, B. (1996). The 20 March, 2000 collapse of the November 1999 dome. MVO Special Report 8, Montserrat Volcano Observatory, 2000.

	(2) {\bf Thompson, G.} (1996). Investigation of the low-frequency seismic wavefield: Justification for a broadband network? MVO Open File Report 96/29, Montserrat Volcano Observatory, 1996.

	(1) {\bf Thompson, G.} (1996). Banded tremor at Soufriere Hills Volcano, July - August 1996. MVO Open File Report 96/28, Montserrat Volcano Observatory, 1996.
\end{comment}
%\newpage

%\section{\sc Awards \& Honours} 
%{\bf NSF - Pending} 
%Add something here like project category and number e.g. AGS 1444401:
%?: Collaborative Research: FlexArray Seismic Imaging of the San Francisco Volcano Field, Arizona
%(\$559,596), PIs: Jochen Braunmiller, Glenn Thompson and Stephen McNutt.

\begin{comment}
\section{\sc Invited Talks}
2003 University of Hawaii (Kona), ``Proposal to install an infrasound network in Montserrat''.\\
2003 University of Hawaii (Honolulu), ``Advances in monitoring at MVO''.\\
2003 Hawaii Volcano Observatory, ``Advances in monitoring at MVO''.\\
2003 Duke University, ``Seismic source mechanisms on Montserrat''.\\
2003 Michigan Tech University, ``Low frequency seismicity at Soufriere Hills Volcano''.\\
2002 EU Volcano Training Network, ``Seismic Monitoring at the Montserrat Volcano Observatory''
2002 EU Volcano Training Network, ``Volcano Seismology''\\
1999 Meteorlogical Research Institute (Tsukuba, Japan), ``Volcano \& earthquake triggering''.\\
\end{comment}

%\section{\sc Conference Activity}
%\\

%\section{\sc Departmental Talks}
%2003 British Geological Survey, ``Monitoring the Soufriere Hills Volcano: The Montserrat Project''\\

\section{\sc Teaching Experience}
{\bf University of South Florida}\\
2018 Fall: Introduction to Programming with MATLAB, Teacher \\
2018 Fall: Time Series Analysis, Co-Teacher \\
2017 Fall: Geovisualization, Co-Teacher\\
2016 Spring: Time Series Analysis, Teacher\\
2015 Spring: Topics in Volcano-Seismology, Teacher\\
2014 Fall: Seismic data analysis with Antelope and MATLAB, Teacher

%{\bf University of Alaska Fairbanks}\\
%Beyond the Mouse, GEOS 692, Co-Teacher, 2009
% with Ronni Grapenthin
%Gave lectures on MATLAB programming covering data import, export and plotting.\\

%{\bf University of Leeds}\\
%Inverse Theory, Teaching Assistant, Fall 1995, 1996, 1997\\
%Assisted Prof. David Gubbins with classes that taught students inverse theory and MATLAB.\\
%PC Skills for Geoscientists, Teaching Assistant, Spring 1995, 1996, 1997\\
%Assisted Dr. Geoff Lloyd with classes that taught students effective use of Microsoft products.\\

\begin{comment}
\section{\sc Research Experience}
Real-time seismic alarm systems capable of notifying scientists of anomalous seismicity (e.g. volcanic tremor, earthquake swarms) that may precede volcanic eruptions.\\
Station state-of-health monitoring systems to automate the removal of "noisy" or "dead" stations from downstream processes including sta/lta triggering, event association and alarm systems.\\
Web-based seismic monitoring systems in Alaska and Montserrat (lead Thompson).\\
Review of seismic monitoring at the Montserrat Volcano Observatory, 2000-2008 (lead Thompson).\\
Banded seismicity at the Soufriere Hills Volcano, 1996-2006, and its relation to dome collapses, explosions and extrusion rate (lead Thompson).\\
Near-real-time determination of pyroclastic flow trajectories and dome growth direction (lead Thompson).\\
A magnitude scale for volcano-seismic events on Montserrat (lead Thompson).\\
Correlation between seismicity, dome collapse volume and ash column height (lead Thompson).\\
Seismicity of the Soufriere Hills Volcano, 2000-2008 (lead De Angelis).\\
Volcanic tremor preceding the March 2009 eruptions of Redoubt Volcano (lead McNutt).\\
Real-time seismic monitoring of the 2009 Redoubt eruption (lead Thompson).\\
Character of the long-period and very-long-period seismic wave field during the 2009 eruption of Redoubt Volcano, Alaska (lead Haney).\\
Tremor analysis of the 2009 Redoubt eruption sequence (lead Haney/Reyes).\\
``Solar coronal heating'', Senior Honours Research Project, University of St. Andrews, Spring 1993.\\
``Simulation of jet-cell muon detectors'', European Center for Nuclear Research (CERN), Summer 1992.\\
``Human evolution'', Rawlins Upper School, Spring 1987.\\
\end{comment}

\begin{comment}
\section{\sc Professional Service}
% sessions chaired if not under conferences, journals reviewed for and dates
Reviewer, Journal of Volcanological and Geothermal Research, 1998-Present\\
Reviewer, Encyclopedia of Volcanoes, 2014\\
%Mentoring of Masters and PhD students working on Montserrat data (2000-2004) and on Alaskan data (2007-present).\\
\end{comment}

%\section{\sc Departmental/University Service}

\begin{comment}
\section{\sc Outreach}
% work with libraries, schools, public lectures etc.
2011-2012 Joy Elementary School Science Night \\
2007-2010 Tanana Valley State Fair, Fairbanks AK\\
%Teaching the public about earthquakes and volcanoes in Alaska\\
2007-2010 Science Pot-Pourri, University of Alaska Fairbanks\\
%Teaching the public about earthquakes and volcanoes in Alaska\\
2006-2009 Seismic lab tours to emergency managers and school groups\\
%Tours of the seismic lab and presentations about seismic hazards, both on- and off-site.\\
2006-2009 Lectures on regional and volcanic seismicity in Alaska to VIPs and school groups\\
2000-2004 Daily tours of MVO to public\\
2000-2003 Tours of MVO to school/university groups and VIPs\\
2000-2003 Advised local authorities on volcano crisis management at weekly meetings\\
2000-2003 Weekly live radio interviews to inform the public about volcanic activity\\
%Gave > 100 tours of the observatory to officials from the European, British and Montserrat governments, and members of the public, including school/university students and tourists.\\
%Gave live radio interviews on Radio Montserrat to explain volcanic and seismic activity, often to inform the public of major activity that was occurring at that moment.\\
%\section{\sc Media Coverage}
%"Montserrat to Host Seismology Workshop", Montserrat Reporter, 23 August 2002, http://www.montserratreporter.org/fra0802-4.htm\\
%"Volcano Hazard, Risk Assessment Next Week", Montserrat Reporter, 12 January 2001, http://www.montserratreporter.org/news0101-2.htm\\
\end{comment}

\begin{comment}
\section{\sc Computer \\ Skills}
Software analysis \& design.\\
Software project management.\\ 
Programming expertise in: MATLAB, Perl, PhP, Python, C/C++, FORTRAN.\\
%Network Administration: Linux (Ubuntu, RedHat, SUSE), Windows, MacOSX.\\
Unix/Linux Administration.\\
%Mapping tools: GMT, GoogleMaps API, KML, ArcGIS.\\
Version control: git/github.com, svn, cvs.\\
Seismic software: Antelope, Earthworm, Seisan, Seislog, GISMO, ObsPy.\\ 
Database schema design \& implementation: MySQL, Antelope.\\
Management of large, real-time geophysical datasets.\\
%Project experience in MATLAB, Perl, PhP, Python, FORTRAN, C, Java, Visual Basic, Tcl/Tk, COBOL, and BBC Basic.
%Scripting expertise with Perl, awk, sed, csh, and bash.\\
%Developed products on Linux (Ubuntu, Red Hat, SuSE), Solaris, Windows (including Cygwin) and MacOS.\\
%%Some experience in J2EE, IDL, GoogleMaps API2, KML, ArcGIS and MySQL.\\
%Automated scripts using the Unix crontab, Windows Scheduler and Antelope rtexec.\\
%Implemented separate operational and development systems with auto-restart and failover capability at Alaska Volcano Observatory.\\
%Extensive experience of developing Antelope applications in Perl, MATLAB, C, PhP and Python.\\
%Developed data acquisition, alarm, analysis and archival systems with failover capabilities at Montserrat Volcano Observatory.\\ 
%Co-admininstered a Linux research computing network at University of Alaska.\\
%Built a Linux/MacOS research computing network at University of South Florida.\\
\end{comment}

\begin{comment}
\section{\sc Research Interests}
Applications of volcano seismology to volcano monitoring.\\
Novel techniques to characterize and locate earthquake swarms, tremor \& dome failure signals.\\
Auto-classification of volcano-seismic signals.\\
Enhancing alarm systems and early warning systems to detect earthquakes, tsunami and volcanic hazards.\\ 
Evolution of seismic monitoring systems.\\ 
Integrated monitoring systems, e.g. utilizing seismic, infrasound, geodetic, lightning and remote sensing techniques.\\
\end{comment}

\begin{comment}
\section{\sc Teaching Areas}
% a list of up to 10 course titles you could teach
Volcano Seismology\\
Volcano Monitoring\\
Computer Programming (in MATLAB, Python, Perl or C)\\
Unix \& shell scripting\\
Seismic Data Analysis\\
Time Series Analysis\\
Digital Signal Processing\\
%Software development with Antelope\\
%Mathematics for Geologists (Calculus, Linear Algebra, Probability, Complex Analysis)\\
\end{comment}

\begin{comment}
%\section{\sc Languages}
%All languages to be listed vertically, with proficiency in reading, speaking, and writing clearly demarcated using terms such as: native, fluent, excellent, conversational, good, can read with dictionary, etc.
%Spanish: conversational, can read
%\\
\end{comment}

\\
%\begin{comment}
\section{\sc Graduate Advisees}
\begin{table}[ht]
\begin{tabular}{c c c c}
\hline\hline
Student & Degree & Role & Dates\\
\hline
Bradford Mack & MS & Advisor & 2018-Present\\
Mitchell Hastings & MS & Committee Member & 2018-Present\\
Daniel Graybeal & MS & Committee Member & 2018-Present\\
Tianyu Rong & MS & Committee Member & 2018-Present\\
Charlie Breithaupt & PhD & Committee Member & 2017-Present\\
Jessica Mejia & PhD & Committee Member & 2017-Present\\
Kathryn Dorn & PhD & Committee Member & 2017-Present\\
Daniel Graybeal & MS & Committee Member & 2017-Present\\
Sajad Jazayeri & PhD & Committee Member & 2017-Present\\
Cassandra Smith	& PhD & Committee Member & 2015-Present\\
Heather McFarlin & PhD & Co-advisor & 2013-Present\\
Cassandra Smith	& MS & Committee Member & 2013-2015\\
Alexandra Farrell & PhD & Co-advisor & 2013-Present\\
Chris Bruton & MS & Committee Member & 2010-2013\\
Helena Buurman & PhD & Committee Member & 2009-2013\\
\hline
\end{tabular}
\label{table:nonlin}
\end{table}
%\end{comment}

\begin{comment}
\section{\sc Product Development}
2015 GISMO - seismic data analysis toolbox for Matlab/Octave. 
2012 Python toolboxes for analysis of earthquake catalogs and event rates.\\
2011 TreMoR: A web-based system for TREmor MOnitoring in Real-time.\\
2010 CWAKE: A Common Workspace for analysis of AlasKan Earthquakes.\\
2010 VOLC2: A Google Maps mash-up for exploring hypocenters in Alaska and Hawaii.\\
2010 MATLAB toolboxes for analysis of earthquake catalogs and event rates (part of GISMO).\\
2009-2010 Earthquake swarm detection and notification system.\\
2007-2008 The Trans-Alaska Pipeline Seismic Monitoring System.\\
2007-2008 Real-time Earthquake Notification for Alaskan Emergency Operations Centers.\\
2006-2007 The Antelope-ShakeMap system.\\
2005 Programmed a robot to scan drill cores (Visual Basic).\\
2005 Developed a system to batch OCR millions of scanned documents from BGS archives.\\
2004 Migrated legacy geophysical software from Silicon Graphics Fortran to Visual Fortran.\\
2001-2002 The MVO web-based monitoring system.\\
2001-2002 System for locating rockfalls and pyroclastic flows in near-real-time.\\
2000-2002 Volcanic tremor and event detection and notification system.\\
1998-2000 IceWeb: a web-based seismic monitoring system for Alaskan volcanoes.\\
\end{comment}

\\
%\begin{comment}
\section{\sc Seismic Fieldwork Leadership}%*}
2018 Montserrat: Seismo-acoustic monitoring of floods/lahars in the Belham Valley\\
2016 Montserrat: Seismic monitoring of geothermal drilling project.\\
2016-2018 Kennedy Space Center, a network of seismo-acoustic stations.\\
2015 Masaya volcano (Nicaragua), two seismic-acoustic stations.\\
%2013 Telica volcano seismic network (Nicaragua), maintenance only.\\
%2012 Alaska seismic network, maintenance only.\\
%2012 Susitna-Watana dam project (Alaska), installation only.\\
%2012 Uturuncu volcano (Bolivia) and Lazufre volcano (Chile), maintenance only.\\
%2010 Bezymianny volcano (Kamchatka), removal only.\\
%2006-2007 Alaska seismic network, installation and maintenance. \\
2000-2004 Montserrat seismic network, installation and maintenance.\\
1996 Soufriere Hills Volcano (Montserrat), installation, maintenance and removal.\\
1995 Mount Batur (Bali), maintenance only.\\
%1995 Taupo Volcanic Zone (New Zealand), installation only.\\
%1994 Stromboli volcano (Italy), maintenance only.\\
%* Listed here is only seismic fieldwork I led. I have participated in other seismic fieldwork throughout Alaska, and at volcanoes in Italy, New Zealand, Kamchatka, Bolivia, Chile and Nicaragua.
%\end{comment}

\begin{comment}
\section{\sc Other Fieldwork Experience}
Seismic network decommission at Bezymianny volcano, Kamchatka, 2010.
Fieldwork in McCarthy, Bering Glacier Camp and Ultima Thule, Alaska, assisting with broadband seismic station installations and repairs, 2006-8.
Hiking trips to Pacaya, Fuego and Santaiguito volcanoes, April 2005.
Ground penetrating radar (Sunderland), seismic refraction (Holme Pierreponte), airborne
geophysics (SW Scotland) and resistivity surveys (Easington) in the UK, 2003-6.
Observation flights, seismic station installation/maintenance, and assistance with GPS, rainwater
and ash sample collection and COSPEC measurements, Montserrat, 2000-4.
GPS survey in along the Parks Highway, Alaska, July 1999.
Installation and data analysis of roving broadband seismic stations, Montserrat, June-August
1996.
Installation, maintenance, and quality control of data from, broadband seismic networks in New
Zealand and Indonesia, 1995.
Gravity, magnetic, seismic refraction & resistivity surveys in Cyprus, March 1994. 
\end{comment}

%\begin{comment}
\section{\sc Professional Training}
% Don't forget there were lots of BGS training courses in Java, J2EE, Oracle/SQL, Excel, Scientific Writing, ...
Earthworm Training Course by Instrumental Software Technologies Inc., 21-25 June 2010.\\
%Intensive Spanish language course, Cuzco, Peru, 2010 (2 weeks).\\
Advanced Scientific Python Programming course, Berlin, Germany, 31 Aug - 4 Sep, 2009.\\
%Intensive Spanish language course, Antigua \& San Pedro, Guatemala, 2005 (3 weeks).\\
%Intensive Spanish language course, Colegio Debiles, Salamanca, Spain, 2004 (4 weeks).\\
%Unleash the Power Within, intensive 4-day leadership workshop, 18-21 February 2005\\
Glowworm Training Course by VDAP at Cascades Volcano Observatory, 9-13 December 2002.\\
Systems analysis and design, TNT Express Worldwide, September-November 1997.\\
CRAC Insight into Management course, University of Durham, 21-26 October 1996.\\ 
%Development of teamwork & communication skills through several 1-3 hour intensive group- oriented case studies including presentations.\\
Conoco Introduction to Management course, University of St. Andrews, 17-18 October 1992.\\ 
%A series of case studies intended to give undergraduates training in teamwork.\\
%Particle physics and particle accelerators, CERN Summer School, Geneva, Switzerland, June- August 1992 (9 weeks).\\
%\end{comment}

%5\begin{comment}
\section{\sc Leadership Experience}
2002-2003 Led the move of MVO to a new purpose built observatory site.\\
2000-2003 Proposed, managed the international bidding process for, and obtained funding for a new digital seismic network for MVO.\\
%2000-2003 Deputy Director of MVO. \\
2000-2003 Seismic Network \& Operations Room Manager of MVO. Managed a team of seismic analysts, field technicians and a software engineer.\\
2000-2002 Recovered, archived and built an online database of all Montserrat seismic data.\\
%1998-Present Led numerous software development \& data management projects. \\ 
%1989 Led a team of four staff at Druck Ltd.\\
%Managed a team of seismic analysts, field technicians and a software engineer in my role as Senior Seismologist, leading the seismic monitoring programme at MVO and the Operations Room.\\
%Led the successful and logistically complex move of the seismic monitoring to a new observatory in 2002 at a time of major activity without losing any data.\\
%Built the modern seismic monitoring programme at MVO, 2000-02.\\
%Filled what was a leadership void at MVO, inspiring, motivating and empathising with staff, and encouraging better teamwork, communication, respect and integrated monitoring.\\
%Had a great rappore with the public on Montserrat, due to my willingness to always make myself available outside of office hours and contribute to community sports.\\
%Staff-student council representation for my M.Sc. course, 1993-4\\
%Managed a team of four staff manufacturing and calibrating pressure transducers at Druck Ltd., 1989.\\
%\end{comment}

\begin{comment}
\section{\sc Hazard Management Experience}
{\bf Alaska Volcano Observatory}\\
Participated in the Redoubt Operations Room, February - April 2009 monitoring the eruption and responding to questions by the media and public.\\
Participated in the Shishaldin Operations Room, March - May 1999.\\
Beeper duty, 1998-1999, 2008-present.\\
{\bf Alaska Earthquake Information Center}\\
Developed real-time notification system for emergency managers.\\
Co-developed the Alaska ShakeMap system.\\
Co-developed the Trans-Alaska Pipeline alarm system.\\
Duty Seismologist, 1998-1999, 2006-2008. Responsible for rapid location and dissemination of information of significant earthquakes in Alaska to authorities, media and public.\\
{\bf Montserrat Volcano Observatory 2000-2004}\\
Member of the Risk Assessment Panel & Scientific Advisory Committee, assessing the hazards and risks posed by the Soufriere Hills Volcano.\\
Day-to-day management of the Operations Room which involved deciding access to the day-time entry and exclusion zones, initiating small scale evacuations, and making live radio broadcasts.\\
Represented MVO at weekly VMSG/VEG meetings with the authorities.\\
Acted as MVO Director on several occasions.\\
Alarm duty 24-7. First responder to > 1000 volcano alarms, with call-down responsibility for local authorities.\\
\end{comment}

%\begin{comment}
%\section{\sc Memberships \& Affiliations}
\section{\sc Memberships}
American Geophysical Union (AGU). Seismological Society of America (SSA). International Association of Volcanology and Chemistry of Earth's Interior (IAVCEI). European Seismological Commission (ESC) Working Group "Seismic phenomena associated with volcanic activity"
\\
%\end{comment}

%\begin{comment}
\section{\sc References}
\begin{multicols}{2}

Richard A. Herd \\ 
Senior Lecturer \\ 
School of Environmental Sciences \\ 
University of East Anglia \\
Norwich, NR4 7TJ \\
United Kingdom \\
Tel: +44 (0)1603 59 3667 \\ 
email: r.herd@uea.ac.uk \\

Silvio De Angelis \\ 
Senior Lecturer \\ 
School of Environmental Sciences \\ 
University of Liverpool \\ 
4 Brownlow Street \\
Liverpool, L69 3GP \\
United Kingdom \\
Tel: +44 (0)151 794 5161 \\
email: S.De-Angelis@liverpool.ac.uk \\


%Stephen R. McNutt \\ 
%Professor \\ 
%School of Geosciences \\ 
%University of South Florida \\ 
%4202 East Fowler Ave., NES107\\ 
%Tampa, FL 33620 \\ 
%Tel: 813-974-4584 \\ 
%email: smcnutt@usf.edu\\ 

%Michael E. West \\ 
%Director \\ 
%Alaska Earthquake Center \\ 
%2156 Koyukuk Dr. \\ 
%University of Alaska, Fairbanks \\ 
%AK 99775-7320 \\ 
%Tel: 907-474-6977 \\ 
%email: mewest@alaska.edu \\

Jackie Caplan-Auerbach \\
Associate Professor \\
Geology Department \\
Western Washington University \\
516 High St., MS 9080 \\
Bellingham, WA 98225 \\
Tel: 360-650-4153 \\
email: caplanj@wwu.edu \\

Paul Friberg \\    
CEO/Seismologist \\
Instrumental Software Technologies, Inc. \\
77 Van Dam Street \\
Saratoga Springs \\
New York, 12866 \\
Tel: Phone 845.256.9290 \\
%Mobile +1.914.489.4888
email: p.friberg@isti.com \\

\end{multicols}

\begin{comment}
\textbf{Capacities in which known:}\\
Richard Herd was Senior Volcanologist at Montserrat Volcano Observatory 2000-2003, while I was Seismic Network Manager.\\
Silvio De Angelis has held many of the same positions as I have: postdoc of Steve McNutt at AVO 2004-2006 (me 1998-2000); Seismic Network Manager at MVO 2006-2009 (me 2000-2003); Staff Seismologist at UAF Geophysical Institute 2009-2012 (me 2006-2013)\\
Jackie Caplan ...
\end{comment}

%John C. Eichelberger, Program Coordinator of the Volcano Hazards Program, USGS, Reston, VA.\\
%William J. C. McCourt, Principle Geologist and Head of International Projects, British Geological Survey, UK.\\
%Simon R. Young, CEO of Caribbean Risk Managers, Washington DC.\\


\end{resume}
\end{document}


