
%Books, Edited Volumes, Refereed Journal Articles, Book Chapters, Conference Proceedings, Book Reviews, Manuscripts in Submission (give journal title), Manuscripts in Preparation, Web-Based Publications, Other Publications (this section can include non-academic publications, within reason). Please note that forthcoming publications ARE included in this section. If they are already in the printing stage, with the full citation and page numbers available, they may be listed the same as other published publications, at the very top since their dates are furthest in the future. If they are in press, they can be listed here with ��in press�� in place of the year.

\begin{comment}
\section{\sc Publications}
%\documentclass[12pt]{article}
%\usepackage[T1]{fontenc}
%\usepackage[utf8]{inputenc}
%\usepackage{lmodern}
%\usepackage[english]{babel}
%\usepackage[autostyle]{csquotes}
%\RequirePackage{doi}
\usepackage[backend=biber,style=phys,sorting=ydnt]{biblatex}
\addbibresource{gtpapers.bib}
\addbibresource{gtbookchapters.bib}
%\title{Peer reviewed journal articles}
%\author{Glenn Thompson}

\begin{document}
\nocite{*}
%\printbibliography[title={Refereed Journal Articles}]
\printbibliography[heading=none]
\\
.\\
Conference papers: 65 (17 as first author)\\ % see gtconferencepapers.tex
Technical reports: 31 (25 as first author)\\ % see gtreports.tex


\end{document}
\nocite{*}
%\printbibliography[title={Refereed Journal Articles}]
\printbibliography[heading=none]
\\
.\\
Conference papers: 68 (17 as first author)\\ % see gtconferencepapers.tex
Technical reports: 31 (25 as first author)\\ % see gtreports.tex

\section{\sc Publications - Book Chapters}

	(2) {\bf Thompson G.} (in press). Seismic Monitoring of Volcanoes. In: Michael Beer, Ioannis A. Kougioumtzoglou, Edoardo Patelli, and Ivan Siu-Kui Au (Eds.) Encyclopedia of Earthquake Engineering. Springer-Verlag Berlin Heidelberg.
	
	(1) McNutt, S.R., {\bf Thompson, G.}, Fee, D., Johnson, J.B., and De Angelis, S. (in press). Seismic and Infrasonic Monitoring, Encyclopedia of Volcanoes, 2nd edition.

\section{\sc Publications - Refereed Journal Articles}

	(17) DeRoin, N., McNutt, S.R., and {\bf Thompson, G.} (2015). Duration-amplitude relationships of volcanic tremor and earthquake swarms preceding and during the 2009 eruption of Redoubt Volcano, Alaska. Journal of Volcanology and Geothermal Research, doi:10.1016/j.volgeores.2015.01.003
	
	(16) McNutt, S.R., {\bf Thompson, G.}, West, M.E., Fee, D., Stihler, S., and Clark, E. (2013). Local seismic and infrasound observations of the 2009 explosive eruptions of Redoubt Volcano, Alaska. Journal of Volcanology and Geothermal Research, 259, 63-76. doi:10.1016/j.jvolgeores.2013.03.016

	(15) Buurman, H., West, M.E., and {\bf Thompson, G.} (2012). The seismicity of the 2009 Redoubt eruption. Journal of Volcanology and Geothermal Research, 259, 16-30. doi:10.1016/j.jvolgeores.2012.04.024

	(14) Schaefer, J.R., ed., (2012), The 2009 eruption of Redoubt Volcano, Alaska, with contributions by Bull, K., Cameron, C., Coombs, M., Diefenbach, A., Lopez, T., McNutt, S., Neal, C., Payne, A., Power, J., Schneider, D., Scott, W., Snedigar, S., {\bf Thompson, G.}, Wallace, K., Waythomas, C., Webley, P., and Werner, C.: Alaska Division of Geological and Geophysical Surveys Report of Investigation 2011-5, 45 p., available at http://www.dggs.alaska.gov/pubs/id/23123.

	(13) Miller, V., Voight, B., Ammon, C., Shalev, E., and {\bf Thompson, G.} (2010). Seismic expression of magma-induced crustal strains and localized fluid pressures during initial eruptive stages, Soufriere Hills Volcano, Montserrat. Geophysical Research Letters, 37(19) 

	(12) {\bf Thompson, G.}, and West, M.E. (2010). Real-time Detection of Earthquake Swarms at Redoubt Volcano, 2009. Seismological Research Letters, 81(3), 505-513. doi:10.1785/gssrl.81.3.505

	(11) Luckett, R., Baptie, B., Ottemoller, L., and {\bf Thompson, G.} (2007). Seismic Monitoring of the Soufriere Hills Volcano, Montserrat. Seismological Research Letters, 78(2), 192-200. doi:10.1785/gssrl.78.2.192

	(10) Taron, J., Elsworth, D., {\bf Thompson, G.}, and Voight, B. (2007). Mechanisms for rainfall-concurrent lava dome collapses at Soufriere Hills Volcano, 2000-2002. Journal of Volcanology and Geothermal Research, 160(1-2), 195-209. doi:10.1016/j.jvolgeores.2006.10.003

	(9) Jaquet, O., Carniel, R., Sparks, S., {\bf Thompson, G.}, Namar, R., and Dicecca, M. (2006). DEVIN: A forecasting approach using stochastic methods applied to the Soufriere Hills Volcano. Journal of Volcanology and Geothermal Research, 153(1-2), 97-111. doi:10.1016/j.jvolgeores.2005.08.013 

	(8) Langer, H., Falsaperla, S., Powell, T., and {\bf Thompson, G.} (2006). Automatic classification and a-posteriori analysis of seismic event identification at Soufriere Hills Volcano, Montserrat. Journal of Volcanology and Geothermal Research, 153(1-2), 110. doi:10.1016/j.jvolgeores.2005.08.012

	(7) Carn, S. A., Watts, R. B., {\bf Thompson, G.}, and Norton, G. E. (2004). Anatomy of a lava dome collapse. Journal of Volcanology and Geothermal Research, 131, 241-264.

	(6) Elsworth, D., Voight, B., {\bf Thompson, G.}, and Young, S. R. (2004). Thermal-hydrologic mechanism for rainfall-triggered collapse of lava domes. Geology, 32(11), 969-972. doi:10.1130/G20730.1

	(5) Edmonds, M., Oppenheimer, C., Pyle, D. M., Herd, R. A., and {\bf Thompson, G.} (2003). SO2 emissions from Soufriere Hills Volcano and their relationship to conduit permeability, hydrothermal interaction and degassing regime. Journal of Volcanology and Geothermal Research, 124(1-2), 23-43. doi:10.1016/S0377-0273(03)00041-6

	(4) Langer, H., Falsaperla, S., and {\bf Thompson, G.} (2003). Application of Artificial Neural Networks for the classification of the seismic transients at Soufriere Hills Volcano, Montserrat. Geophysical Research Letters, 30(21), 1-5. doi:10.1029/2003GL018082

	(3) Matthews, A. J., Barclay, J., Carn, S., {\bf Thompson, G.}, Alexander, J., Herd, R., and Williams, C. (2002). Rainfall-induced volcanic activity on Montserrat. Geophysical Research Letters, 29(13). doi:10.1029/2002GL014863

	(2) Jolly, A. D., {\bf Thompson, G.}, and Norton, G. (2002). Locating pyroclastic flows on Soufriere Hills Volcano, Montserrat, West Indies, using amplitude signals from high dynamic range instruments. Journal of Volcanology and Geothermal Research, 118(3-4), 299-317. doi:10.1016/S0377-0273(02)00299-8

	(1) {\bf Thompson, G.}, McNutt, S. R., and Tytgat, G. (2002). Three distinct regimes of volcanic tremor associated with the eruption of Shishaldin Volcano, Alaska 1999. Bulletin of Volcanology, 64(8), 535-547. doi:10.1007/s00445-002-0228-z


\section{\sc Publications - Peer-reviewed Open-File Reports}

	(3) Gunn, D. A., Jackson, P. D., {\bf Thompson, G.} (2005), Fine Scale Structure of Core by Novel Resistivity Imaging. BGS Report No. IR/05/065.

	(2) Gunn, D. A., Nelder L. M., Pearson, S., {\bf Thompson, G.}, Carney, J. N., Wallis, H., (2005), Investigating shear wave methods to characterize sand and gravel thicknesses at Holme Pierrepont, Nottingham. BGS Report No. IR/03/116.

	(1) Cuss, R. J., {\bf Thompson, G.}, (2003), Ground penetrating radar investigation of the fissuring of the A690 in Houghton-le-Spring , Commissioned Report CR/03/301R, Urban Geoscience and Geological Hazards Programme, British Geological Survey.
\end{comment}
\begin{comment}	
\section{\sc Theses}

	{\bf Thompson, G.}, (1999), Modelling of seismo-volcanic sources. Ph.D. Thesis, University of Leeds, UK.

	{\bf Thompson, G.}, (1994), Forward modelling of bottom-simulating reflectors at the Makran accretionary prism. M.Sc. Dissertation, University of Durham, UK.
\end{comment}

\begin{comment}
\section{\sc Publications - Conference Abstracts}

	%%George, O., Latchman, J.L., {\bf Thompson, G.}, Connor, C.B. and Malservisi, R. (2015). Improving Seismic Event Locations on Dominica Using the HypoDD algorithm to Enhance Volcanic Hazard Assessments, ... 2015.

	(47) {\bf Thompson, G.}, (2014). Towards a Comprehensive Catalog of Volcanic Seismicity. EOS Trans., AGU., ..., Fall Meet. Suppl., Abstract ..., December 2014.

	(46) Smith, C., McNutt, S.R, and {\bf Thompson, G.} (2014). Explosion Quakes: The 2007 Eruption of Pavlof. EOS Trans., AGU., ..., Fall Meet. Suppl., Abstract ..., December 2014.

	(45) McFarlin, H.L, Christensen, D.H., {\bf Thompson, G.}, McNutt, S.R., Ryan, J.C., Ward, K.M., Zandt, G., and West, M.E. (2014). Receiver Function Analyses of Uturuncu Volcano (Bolivia) and Lastarria/Cordon Del Azufre Volcanoes (Chile). EOS Trans., AGU., ..., Fall Meet. Suppl., Abstract ..., December 2014.

	(44) {\bf Thompson, G.}, and McNutt, S.R. (2014). Banded tremor at Soufriere Hills Volcano, Montserrat. Seismo. Soc. Am. Annual Meeting, Anchorage, 30 April - 2 May 2014.

	(43) McFarlin, H., Christensen, D., and {\bf Thompson, G.} (2014). Receiver Function Analyses of Uturuncu Volcano, Bolivia. Seismo. Soc. Am. Annual Meeting, Anchorage, 30 April - 2 May 2014.

	(42) {\bf Thompson, G.}, West, M. E., (2010), Real-time Tracking of Earthquake Swarms at Redoubt Volcano, 2009. Seismo. Soc. Am. Annual Meeting, Portland, 21-23 April 2010.

	(41) {\bf Thompson, G.}, West, M. E., (2009), Alarm systems detect volcanic tremor and earthquake swarms during Redoubt eruption, 2009. EOS Trans., AGU, 90 (52), Fall Meet. Suppl., Abstract V43A-2215, December 2009.

	(40) Miller, V., Ammon, C., Voight, B., and {\bf Thompson, G.} (2006). Precise hypocenter location of high-frequency-onset earthquakes, during the initial stages of activity at Soufriere Hills Volcano, Montserrat. AGU Fall Meeting Abstracts, 1, 0586. 

	(39) Elsworth, D., Voight, B., Taron, J., {\bf Thompson, G.}, Vinciguerra, S., and Simmons, J. (2005). Gravitational collapse of lava domes triggered by volcanic fluids. AGU Fall Meeting Abstracts, 1, 07. 

	(38) Mattioli, G.S., Voight, B., Linde, A.T., Sacks, I.S., Watts, P., Hidayat, D., Young, S.R.,  Widiwijayanti, C., Shalev, E., Malin, P.E., and {\bf Thompson, G.} (2005). The CALIPSO borehole project at Soufriere Hills Volcano, Montserrat, BWI: Status and scientific overview of prodigious dome collapse of July 2003. AGU Spring Meeting Abstracts, 1, 05. 

	(37) Voight, B., Mattioli, G.S., Linde, A.T., Sacks, I.S., Young, S.R., Malin, P., Shalev, E., Hidayat, D., Elsworth, D., Widiwijayanti, C., Miller, V., McWhorter, N., Schleigh, B., Johnston, W., Sparks, R.S.J., Neuberg, J., Bass, V., Dunkley, P., Herd, R., Jolly, A.D., Norton, G., Syers, T., {\bf Thompson, G.}, Williams, P., Williams, D., Clarke, A., CALIPSO Borehole Monitoring Project at Soufriere Hills Volcano, Montserrat, BWI: Overview, and Response of Magma Reservoir to Prodigious Dome Collapse. Scientific conference, Montserrat, July 2005.

	(36) Miller V., Ammon, C., Voight, B., {\bf Thompson, G.}, Precise hypocenter location of high-frequency-onset earthquakes and changing stress conditions beneath Soufriere Hills volcano, Montserrat. Scientific conference, Montserrat, July 2005.

	(35) Voight, B., Mattioli, G., Linde, A., Sacks, I., Young, S., Malin, P., Shalev, E., Hidayat, D., Elsworth, D., Widiwijayanti, C., Miller, V., McWhorter, N., Schleigh, B., Johnston, W., Sparks, R., Neuberg, J., Bass, V., Dunkley, P., Herd, R., Jolly, A., Norton, G., Syers, T., {\bf Thompson, G.}, Williams, C., Williams, D., Clarke, A. (2005). CALIPSO Borehole Monitoring Project at Soufriere Hills Volcano, Montserrat, BWI: Overview, and Response of Magma Reservoir to Prodigious Dome Collapse. Soufrière Hills Volcano -- Ten-Years On Scientific Conference. (Jul 2005).

	(34) {\bf Thompson, G.} (2005). Advances in seismic monitoring at the Montserrat Volcano Observatory 2000-2003. Soufrière Hills Volcano -- Ten-Years On Scientific Conference, July 2005.

	(33) Mattioli, G.S., Young, S.R., Linde, A.T., Sacks, I.S., Malin, P.E., Shalev, E., Hidayat, D., Elsworth, D., Widiwijayanti, C., Miller, V., and {\bf Thompson, G.} (2004). Calipso borehole instrumentation at a Bezymianny-like andesite Volcano: Calipso project at Soufriere Hills Volcano, Montserrat., 82. 

	(32) Mattioli, G., Voight, B., Young, S., Linde, A., Sacks, I., Malin, P., Shalev, E., Hidayat, D., Elsworth, D., Widiwijayanti, C., Miller, V., McWhorter, N., Schleigh, B., Johnston, W., Sparks, R., Neuberg, J., Bass, V., Dunkley, P., Herd, R., Jolly, A., Norton, G., Syers, T., {\bf Thompson, G.}, Williams, P., Williams, D., Clarke, A. (2004). CALIPSO project at Soufriere Hills Volcano, Montserrat: Prototype PBO instrumentation installed and captures massive dome collapse of July 2003. IAVCEI General Assembly. (Nov 2004).

	(31) Miller, V., {\bf Thompson, G.}, Voight, B., and Ammon, C. (2004). Precise hypocenter location of high-frequency-onset earthquakes, tomography, and changing stress conditions beneath Soufriere Hills Volcano, Montserrat. AGU Fall Meeting Abstracts, 1, 05. 

	(30) Shannon, J., G. Bluth, G. Ryan, A.D. Jolly, G. Thompson, and M. Edmonds (2004). Short-term fluctuations in SO2 emission from automated DOAS measurement on Montserrat, IAVCEI, Pucon, Chile, 2004.

	(29) Voight B., Mattioli G.S., Linde A.T., Sacks I.S., Watts P., Hidayat D., Young S.R., Widiwijayanti C., Shalev E., Malin P.E., Elsworth D., Williams P., Van Boskirk E., Thompson G., Syers T., Sparks R.S.J., Schleigh B., Norton G., Neuberg J., Miller V., McWhorter N., Johnston W., Dunkley P., Clarke A.B., Bass V., Collapse of lava dome on Montserrat, July 2003 with unique CALIPSO geophysical measurement of pyroclastic flows and PF-generated tsunami waves, IAVCEI General Assembly, Pucon, Chile, 2004.

	(28) Langer, H., S. Falsaperla, T. Powell and G. Thompson (2004). Reclassification of seismic transients at Soufriere Hills Volcano, Montserrat, and applications of Artificial Neural Networks. EGU General Assembly, Nice 2004, EGU04-A-04026, April 2004.

	(27) Voight, B., Mattioli, G.S., Linde, A.T., Sacks, I.S., Young, S.R., Malin, P.E., Shalev, E., Hidayat, D., Elsworth, D., Widiwijayanti, C., and {\bf Thompson, G.} (2004). CALIPSO borehole monitoring project at Soufriere Hills Volcano, Montserrat, BWI: Overview, and response of magma reservoir to prodigious dome collapse. AGU Fall Meeting Abstracts, 1, 03. 

	(26) Shannon, J., G. Bluth, M. Edmonds, and G. Thompson (2003). Volcanic SO2 emissions vs. seismicity - July 2002 LP swarm. EOS Trans., AGU, 84 (46), Fall Meet. Suppl., Abstract V52B-0435, December 2003. 

	(25) Dunkley, P. Voight, B., Edmonds, M., Herd, R., Strutt, M., {\bf Thompson, G.}, Bass, V., Aspinall, W.P., Neuberg, J., and Sparks, R. (2003). The rise and fall of the Soufrière Hills Volcano lava dome, Montserrat, BWI, July 2001-July 2003: Science, hazards, and volatile public perceptions. EOS Trans., AGU, 84 (46), Fall Meet. Suppl., Abstract V51F-0342

	(24) Hidayat, D., Voight, B., Mattioli, G., Young, S.R., Linde, A.T., Sacks, I.S., Malin, P.E., Shalev, E., Elsworth, D., Widiwijayanti, C., and {\bf Thompson, G.} (2003). Seismo-acoustics, VLP and ULP signals, and other comparisons of surface broadband and CALIPSO borehole data at Soufriere Hills Volcano, Montserrat, BWI. EOS Trans., AGU, 84 (46), Fall Meet. Suppl.,

	(23) Voight, B., G.S. Mattioli, S.R. Young, A.T. Linde, I.S. Sacks, P.E. Malin, E. Shalev, D. Hidayat, D. Elsworth, C. Widiwijayanti, V. Miller, R.S.J. Sparks, J. Neuberg, V. Bass, P. Dunkley, M. Edmonds, R. Herd, A. Jolly, G. Norton, and G. Thompson (2003).  CALIPSO Borehole Instrumentation Project at Soufriere Hills volcano, Montserrat, BWI: Overview and prospects. EOS Trans., AGU, 84 (46), Fall Meet. Suppl., Abstract V51J-0409, 2003.

	(22) Young, S.R., B. Voight, G.S. Mattioli, A.T. Linde, I.S. Sacks, P.E. Malin, E. Shalev, D. Hidayat, D. Elsworth, R.S. Sparks, J. Neuberg, P.N. Dunkley, G. E. Norton, R.A. Herd, M. Edmonds, G. Thompson, A. Jolly and V. Bass, 2003. Linking surface activity to the deep volcanic plumbing sustem: the CALIPSO borehole observatory project on Montserrat. EOS Trans., AGU, 84(46), Fall Meet. Suppl., Abstract V51J-0409, 2003.

	(21) Langer H., Falsaperla S., Powell T., {\bf Thompson, G.} (2003). Classification of seismic transients recorded on Soufriere Hills volcano, Montserrat, using Artifical Neural Networks. 13th ESC-WG Annual meeting, Pantelleria, Italy, 23-28 September 2003.

	(20) Langer, H., Falsaperla, S., and {\bf Thompson, G.} (2003). Automatic identification of seismic transients recorded on Soufriere Hills Volcano, Montserrat. EGS-AGU-EUG Joint Assembly, 1. pp. 3531. 

	(19) Neuberg, J., Edmonds, M., Herd, R., {\bf Thompson, G.}, Green, D., Powell, T., and Jolly, A. (2003). Multi-parameter Monitoring on Montserrat: Concepts, Megabytes of Data and Results. IUGG General Assembly, Sapporo, Japan, 30 June - 11 July 2003.
	
	(18) Young, S., Voight, B., Edmonds, M., Herd, R., {\bf Thompson, G.}, and Mattioli, G. (2003). Investigation of cyclic activity at Soufriere Hills volcano, Montserrat, through multiple monitoring techniques. IUGG General Assembly, Sapporo, Japan, 30 June - 11 July 2003.

	(17) Jolly A.D., Herd, R. A., Norton, G. E., {\bf Thompson, G.}, Bass, V. A. (2003). Location and duration-size distribution of dome failure events at the Soufriere Hills Volcano, Montserrat, West Indies. IUGG General Assembly, Sapporo, Japan, 30 June - 11 July 2003.

	(16) {\bf Thompson, G.}, Dunkley, P.N., Edmonds, M., Herd, R. A. (2003). Recent advances in volcano monitoring at the Montserrat Volcano Observatory, Seismo. Soc. Am. Annual Meeting, Puerto Rico, 29 April - 3 May, 2003. 

	(15) {\bf Thompson, G.} (2002). Volcano-seismic monitoring of the Soufriere Hills Volcano, Montserrat. 12th ESC-WG Annual Meeting, Montserrat, September 2002. 

	(14) Elsworth, D., Voight, B., Calder, E.S., Edmonds, M., Herd, R., Norton, G., Syers, T., {\bf Thompson, G.}, Watts, R., and Young, S.R. (2002). Some Models of Rainfall-triggered Collapse of Lava Domes in Potentially Gas-effusive Environments, May 2002.

	(13) Voight, B., Young, S.R., Baptie, B.J., Bass, V., Duffell, H., Dunkley, P.N., Edmunds, M., Herd, R.A., Jolly, A.D., Norton, G. and {\bf Thompson, G.} (2001). Multiparameter measurements at Montserrat and their interpretation: Honoring the memory of Bruno Martinelli. AGU Fall Meeting Abstracts, 1, 02. 

	(12) Jolly, A.D., {\bf Thompson, G.}, and Norton, G.E. (2001). Locating Pyroclastic Flows on Soufriere Hills Volcano, Montserrat, West Indies, Using Amplitude Signals From High Dynamic Range Instruments. EOS Trans., AGU, 82(47), December 2001.

	(11) Voight, B., Young, S., Baptie, B. Bass, V., Duffell, H., Dunkley, P., Edmonds, M., Herd, R., Jolly, A., Norton, G., Syers, T., {\bf Thompson, G.}, Williams, C., Williams, D. (2001). EOS Trans., AGU, 82(47), Fall Meet. Suppl., 2001.

	(10) {\bf Thompson, G.}, and Jolly, A.D. (2001). Detecting switches in dome growth direction by mapping rockfall activity at Soufriere Hills Volcano, Montserrat, 11th ESC-WG Annual meeting, Tenerife, September 2001.
	
	(9) {\bf Thompson, G.}, Powell, T.W., and Neuberg, J. (2001). Cataloguing banded tremor episodes at the Soufriere Hills Volcano, Montserrat, 1996-2001. EU Workshop on Calderas, Vesuvius Volcano Observatory, Italy, May 2001. 

	(8) Jolly, A. D., Young, S. R., Cabey, L., {\bf Thompson, G.} (1999). The Reduced Displacement of large explosive eruptions at Soufriere Hills volcano, Montserrat, West Indies, using broadband seismic data. EOS Trans., AGU, 80, 1145, December 1999

	(7) {\bf Thompson, G.}, McNutt, S. R., Mann, D., Bower, G. R., (1999). Monitoring and analysis of volcanic tremor reduced displacement and spectra associated with eruptions of Shishaldin Volcano, April 1999. EOS Trans., AGU, December 1999. 

	(6) {\bf Thompson, G.}, Benoit, J., Lindquist, K., Hansen, R., and McNutt, S. (1998). Near-realtime WWW-based monitoring of Alaskan Volcanoes: The IceWeb system. EOS, Transactions, American Geophysical Union, 79(45), F957. 

	(5) {\bf Thompson, G.} and Neuberg, J. (1998). Amplitude modelling of long period seismic phases at Stromboli and estimation of source parameters. EOS Trans., AGU, December 1998.

	(4) {\bf Thompson, G.} and Neuberg, J. (1997). Wavenumber-frequency modelling of long period seismic phases at Stromboli Volcano. 7th ESC-WG Annual Meeting, Ambleside, UK, September 1997. 

	(3) Neuberg J., {\bf Thompson, G.}, and Luckett, R. (1996). Models for Strombolian eruptions inferred from broadband data. Annales Geophysicae, Supplement 1 to Vol 14, C281, 1996.

	(2) {\bf Thompson, G.} (1996). Banded tremor at Montserrat, July-August 1996, Volcanic Studies Group, The Geological Society, Burlington House, London UK, 27 November 1996.

	(1) Neuberg, J., and {\bf Thompson, G.}, (1995). A model of Strombolian eruptions inferred from seismic broadband data. EOS Trans., AGU, 76, 46, 658, December 1995.
\end{comment}

%\section {\sc Publications - Submitted}

%\section {\sc Publications - In Preparation}\\

\begin{comment}
\section{\sc Publications - Technical \& Scientific Reports}

	(27) {\bf Thompson, G.} (2013). Web-based seismic spectrograms at the GI. GISEIS Technical Whitepaper 2013-03, Geophysical Institute, University of Alaska Fairbanks.

	(26) {\bf Thompson, G.} (2013). AEIC Web Presence Project. GISEIS Technical Whitepaper 2013-02, Geophysical Institute, University of Alaska Fairbanks.

	(25) {\bf Thompson, G.} (2013). Mirroring the AVO event catalog from AQMS to Antelope. GISEIS Technical Whitepaper 2013-01, Geophysical Institute, University of Alaska Fairbanks.

	(24) {\bf Thompson, G.} (2011). Optimizing detection of earthquakes swarms in the presence of noise, using Antelope. GISEIS Technical Whitepaper 2011-01, Geophysical Institute, University of Alaska Fairbanks.

	(23) {\bf Thompson, G.} (2009). IceWeb2. GISEIS Technical Whitepaper 2009-01, Geophysical Institute, University of Alaska Fairbanks.

	(22) {\bf Thompson, G.} Robinson, M., (2008). Configuring an iMac as an Earthquake Notification System. GISEIS Technical Whitepaper 2008-06, Geophysical Institute, University of Alaska Fairbanks.

	(21) {\bf Thompson, G.} (2008). User Guide to the iMac Earthquake Notification. GISEIS Technical Whitepaper 2008-05, Geophysical Institute, University of Alaska Fairbanks.

	(20) {\bf Thompson, G.} (2008). Delivery of Earthquake Notification Systems to Emergency Managers in Alaska. GISEIS Technical Whitepaper 2008-04, Geophysical Institute, University of Alaska Fairbanks.

	(19) {\bf Thompson, G.}, Robinson, M. (2008). The Trans-Alaska Pipeline ShakeMap system. GISEIS Technical Whitepaper 2008-03, Geophysical Institute, University of Alaska Fairbanks.

	(18) {\bf Thompson, G.}, Martirosyan, A., Robinson, M., Hansen, R. (2008). The AEIC ShakeMap system. GISEIS Technical Whitepaper 2008-02, Geophysical Institute, University of Alaska Fairbanks.

	(17) {\bf Thompson, G.}, La Fevers, M. (2008). Revised procedures for maintaining the AEIC master stations database. GISEIS Technical Whitepaper 2008-01, Geophysical Institute, University of Alaska Fairbanks.

	(16) {\bf Thompson, G.}, Robinson, M. (2007). The AEIC Antelope-QDDS Interface. GISEIS Technical Whitepaper 2007-02, Geophysical Institute, University of Alaska Fairbanks.

	(15) {\bf Thompson, G.}, La Fevers, M. (2007). Maintaining the AEIC master stations database. GISEIS Technical Whitepaper 2007-01, Geophysical Institute, University of Alaska Fairbanks.
	
	(14) {\bf Thompson, G.} (2004). Rebuilding MVO seismic monitoring, January - February 2004. MVO Open File Report 04/04, Montserrat Volcano Observatory, 2004.

	(13) Dunkley, P.N., Norton, G.E., Hards, V.L., Ryan, G.A., Jolly, A.D., {\bf Thompson, G.} (2004). Summary of volcanic activity from May 2003 to February 2004. MVO Open File Report 04/02, Montserrat Volcano Observatory, 2004.

	(12) {\bf Thompson, G.} (2003). The role of the Software Engineer within the seismic monitoring programme. MVO Open File Report 03/03, Montserrat Volcano Observatory, 2003.

	(11) {\bf Thompson, G.} (2003). Moving the seismic monitoring from Mongo Hill to Flemings. MVO Open File Report 03/02, Montserrat Volcano Observatory, 2003.

	(10) Dunkley, P.N., Edmonds, M, Herd, R.A., {\bf Thompson, G.} (2003). Summary of volcanic activity and monitoring data, with particular emphasis on the second phase of dome building November 1999 to June 2003. MVO Open File Report 03/01, Montserrat Volcano Observatory, 2003.

	(9) {\bf Thompson, G.} (2002). Seismic software at MVO, January 2002. MVO Open File Report 02/01, Montserrat Volcano Observatory, 2002.

	(8) {\bf Thompson, G.}, McNutt, S.R. (2002). Near-real-time web-based volcano-seismic monitoring: The IceWeb system. GISEIS Technical Whitepaper 2002-01, Geophysical Institute, University of Alaska Fairbanks.

	(7) {\bf Thompson, G.} (2001). An overview of banded tremor at the Soufriere Hills Volcano, 1996-2001. MVO Open File Report 01/02, Montserrat Volcano Observatory, 2001.

	(6) {\bf Thompson, G.}, Dunkley, P.N., Edmonds, M., Herd, R.A., Jolly, A.D. (2001). The 29 July 2001 collapse. MVO Special Report 9, Montserrat Volcano Observatory, 2001.

	(5) {\bf Thompson, G.} (2000). Upgrading the MVO seismic data acquisition and analysis systems and collaborating with SRU, October 2000. MVO Open File Report 00/03, Montserrat Volcano Observatory, 2000.

	(4) {\bf Thompson, G.} (2000). Review of MVO Seismic Monitoring, August 2000. MVO Open File Report 00/02, Montserrat Volcano Observatory, 2000. 

	(3) {\bf Thompson, G.}, Watts, R., Carn, S., Norton, G. Nye, R., Voight, B. (1996). The 20 March, 2000 collapse of the November 1999 dome. MVO Special Report 8, Montserrat Volcano Observatory, 2000.

	(2) {\bf Thompson, G.} (1996). Investigation of the low-frequency seismic wavefield: Justification for a broadband network? MVO Open File Report 96/29, Montserrat Volcano Observatory, 1996.

	(1) {\bf Thompson, G.} (1996). Banded tremor at Soufriere Hills Volcano, July - August 1996. MVO Open File Report 96/28, Montserrat Volcano Observatory, 1996.
\end{comment}